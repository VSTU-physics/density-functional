\documentclass[article]{ncc}
\usepackage[russian]{babel}
\usepackage[utf8]{inputenc}
\usepackage{amsmath}
\renewcommand{\vec}{\boldsymbol}
\title{Теория функционала плотности}
\author{Абдрахманов В.}
\begin{document}
    \maketitle
    \tableofcontents
    \section{Уравнения Кона-Шема}
    Согласно теории Хоэнберга-Кона, полная энергия системы является функционалом электронной плотности:
    \begin{equation}
        E[n] = T[n] + V[n].
    \end{equation}
    Основная идея состоит в представлении кинетической энергии в виде суммы:
    \begin{equation}
        T[n] = T_s[n] + V_H[n] + E_{xc}[n],
    \end{equation}
    \begin{equation}
        T_s = \int \psi^*(\vec{r}) \left(-\frac{\nabla^2}{2}\right) \psi(\vec{r})\,d^3\vec{r},
    \end{equation}
    \begin{equation}
        V_H = \frac{1}{2}\int\int \frac{n(\vec{r})n(\vec{r}')}{|\vec{r}' - \vec{r}|}\,d^3\vec{r}\,d^3\vec{r}',
    \end{equation}
    Энергия основного состояния является минимумом этого функционала. Проварьируем по электронной плотности:
    \begin{align}
        \delta E = & \int \delta\psi^*(\vec{r}) \left(-\frac{\nabla^2}{2}\right) \psi(\vec{r})\,d^3\vec{r} + \int \psi^*(\vec{r}) \left(-\frac{\nabla^2}{2}\right) \delta\psi(\vec{r})\,d^3\vec{r} +\\
        + & \int\int \frac{n(\vec{r}')}{|\vec{r}' - \vec{r}|}\,d^3\vec{r}'\delta n\,d^3\vec{r} + \int v_{xc} \delta n\,d^3\vec{r} + \int v_{ext} \delta n\,d^3\vec{r}.
    \end{align}
    \begin{equation}
        n = \psi\psi^*,\quad \delta n = \psi\delta\psi^* + \psi^*\delta\psi
    \end{equation}
    Откуда взять множитель Лагранжа?
\end{document}