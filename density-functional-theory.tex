\documentclass[11pt,russian]{ncc}
\ToCenter[h]{160mm}{240mm}
\usepackage[headings]{ncchdr}
\sectionstyle{center}
\indentaftersection
\usepackage[T2A]{fontenc}
\usepackage[utf8]{inputenc}
\usepackage[russian]{babel}
\makeatletter
\let \@Asbuk\russian@Alph
\let \@asbuk\russian@alph
\makeatother
\usepackage{amsmath,amssymb}
\usepackage{physics}
\usepackage[square, numbers, sort&compress]{natbib}
\usepackage[colorlinks]{hyperref}
\newcommand{\eps}{\varepsilon}
\renewcommand{\phi}{\varphi}
\renewcommand{\vec}{\boldsymbol}
\title{Теория функционала плотности}
\author{Абдрахманов В.}
\begin{document}
    \maketitle
    \tableofcontents
    \section*{Обозначения}
    В этом тексте используется атомная система единиц: $ \hbar = e = m_e = 1$
    \begin{description}
        \item[$n(\vec{r})$] электронная плотность
    \end{description}
    \section{Основные положения}
    \subsection{Теория Хоэнберга-Кона}
    В своей работе \cite{Hohenberg-Kohn} сформулировали 2 теоремы, на которых построена вся теория функционала плотности.

    \begin{theorem}
        Для любой системы, находящейся во внешнем потенциале \( V(\vec{r}) \), этот внешний потенциал является (однозначным) функционалом электронной плотности основного состояния.
    \end{theorem}

    \begin{theorem}
        Энергия системы является функционалом электронной плотности. Энергия основного состояния является наименьшим значением этого функционала.
    \end{theorem}

    См. также:
    \begin{enumerate}
        \item \url{http://www.physics.metu.edu.tr/~hande/teaching/741-lectures/lecture-06.pdf}
    \end{enumerate}
    \subsection{Уравнения Кона-Шема}
    Согласно теории Хоэнберга-Кона, полная энергия системы является функционалом электронной плотности:
    \begin{equation}
        E[n] = T[n] + U_{ee}[n] + V[n].
    \end{equation}
    Основная идея состоит в представлении кинетической энергии в виде суммы:
    \begin{equation}
        T[n] = T_s[n] + T_c[n],
    \end{equation}
    где \( T_s \) -- это кинетическая энергия системы невзаимодействующих электронов, а \( T_c \) содержит корреляционную часть кинетической энергии. Для \( T_s \) мы можем в явном виде выписать выражение
    \begin{equation}
        T_s = \int \Psi^*(\vec{r}_1, \ldots, \vec{r}_N) \left(-\frac{\nabla^2}{2}\right) \Psi(\vec{r}_1, \ldots, \vec{r}_N)\,d^3\vec{r}.
    \end{equation}
    Многоэлектронная волновая функция имеет вид
    \begin{equation}
        \Psi(\vec{r}_1, \ldots, \vec{r}_N) = \frac{1}{\sqrt{N!}}\det \phi_i(\vec{r}_j),
    \end{equation}
    откуда
    \begin{equation}
        T_s = \sum_{i}^N\expval{-\frac{\nabla^2}{2}}{\phi_i},
    \end{equation}
    \begin{equation}
        n(\vec{r}) = \sum_{i}^N|\phi_i(\vec{r})|^2.
    \end{equation}
    Функционал взаимодействия электронов можно представить в виде
    \begin{equation}
        U_{ee} = V_H + V_x,
    \end{equation}
    \begin{equation}
        V_H = \frac{1}{2}\iint \frac{n(\vec{r})n(\vec{r}')}{|\vec{r}' - \vec{r}|}\,d^3\vec{r}\,d^3\vec{r}',
    \end{equation}
    Для учёта корреляции и обмена не существует нормального выражения, поэтому функционал энергии представляют в виде
    \begin{equation}
        E[n] = T_s[n] + V[n] + V_H[n] + E_{xc}[n],
    \end{equation}
    где последнее слагаемое учитывает обмено-корреляционные эффекты и выбирается от балды.

    Минимизировать непосредственно функционал можно, но не очень удобно. Поэтому проварьируем его и получим уравнения Кона-Шема. Для этого представим функционал электронной плотности как функционал, зависящий от \( N \) функций \(\phi_i\) с \( N \) дополнительными условиями \( \bra{\phi_i}\ket{\phi_i} = 1 \). Метод множителей Лагранжа приводит нас к уравнению на собственные значения
    \begin{equation}
        \left(-\frac{\nabla^2}{2} + v(\vec{r}) + v_H(n;\vec{r}) + \eps_{xc}(n;\vec{r})\right)\phi = \eps\phi.
    \end{equation}

    Так как перед нами стоит задача отыскания электронной плотности основного состояния, то нас инересуют наименьшие собственные значения и соответствующие им собственные функции.

    \subsection{Представление плоских волн}
    Для периодических систем волновая функция будет иметь вид блоховской функции:
    \begin{equation}
        \phi_{n\vec{k}} = e^{i\vec{k}\cdot\vec{r}} u_{n\vec{k}}(\vec{r}),
    \end{equation}
    где функция \(u_{n\vec{k}}(\vec{r})\) это периодическая функция с периодом элементарной ячейки. Она может быть представлена в виде ряда Фурье:
    \begin{equation}
        u_{n\vec{k}}(\vec{r}) = \sum_{\vec{G}} c_{n\vec{k}}(\vec{G}) e^{i\vec{G}\cdot\vec{r}}.
    \end{equation}
    Подставляя в уравнение, получаем
    \begin{equation}
        \left(-\frac{\nabla^2}{2} + v(\vec{r}) + v_H(n;\vec{r}) + \eps_{xc}(n;\vec{r})\right)\sum_{\vec{G}'} c_{n\vec{k}}(\vec{G}') e^{i(\vec{G}' + \vec{k})\cdot\vec{r}} = \eps\sum_{\vec{G}'} c_{n\vec{k}}(\vec{G}') e^{i(\vec{G}' + \vec{k})\cdot\vec{r}}.
    \end{equation}
    Домножим на \( e^{-i(\vec{G}+\vec{k})\cdot\vec{r}} \) и проинтегрируем по ячейке:
    \begin{equation}
        \frac{G^2+k^2}{2}  c_{n\vec{k}}(\vec{G}) + \frac{1}{\Omega}\int_\Omega \left(v(\vec{r}) + v_H(n;\vec{r}) + \eps_{xc}(n;\vec{r})\right)\sum_{\vec{G}'} c_{n\vec{k}}(\vec{G}') e^{i(\vec{G}' - \vec{G})\cdot\vec{r}} d\vec{r} = \eps c_{n\vec{k}}(\vec{G})
    \end{equation}
    \begin{equation}
        \frac{G^2+k^2}{2}  c_{n\vec{k}}(\vec{G}) + \frac{1}{\Omega}\sum_{\vec{G}'}\left[\tilde{v}(\vec{G}-\vec{G}') + \tilde{v}_H(n;\vec{G} - \vec{G}') + \tilde{\eps}_{xc}(n;\vec{G} - \vec{G}')\right] c_{n\vec{k}}(\vec{G}') = \eps c_{n\vec{k}}(\vec{G})
    \end{equation}
    Теперь перед нами встаёт задача определения коэффициентов Фурье потенциала решётки, потенциала Хартри и обменно-корреляционного потенциала.

    Рассмотрим сначала потенциал решётки. Для простоты будем считать, что в элементарной ячейке находится только 1 атом, радиус-вектор которого в системе координат его ячейки \( \vec{\xi} \). Определим Фурье-коэффициенты:
    \begin{equation}
        \tilde{v}(\vec{G}) = \int_\text{cell} v(\vec{r}) e^{-i\vec{G}\cdot\vec{r}}\,d\vec{r} = Z \int_\text{cell} \sum_{i} \frac{e^{-i\vec{G}\cdot\vec{r}}}{|\vec{r} - \vec{r}_i|}\,d\vec{r}.
    \end{equation}
    Обозначим \(\vec{R}_i\) начало координат i-ой ячейки, тогда \( \vec{r}_i = \vec{R}_i + \vec{\xi} \) и выполняя замену переменного, получаем
    \begin{equation}
        \tilde{v}(\vec{G}) = Z \sum_{i} \int_\text{i-cell} \frac{e^{-i\vec{G}\cdot(\vec{r} - \vec{R}_i)}}{|\vec{r} - \vec{\xi}|}\,d\vec{r} =
        Z \int_{\mathbb{R}^3}\frac{e^{-i\vec{G}\cdot\vec{r}}}{|\vec{r} - \vec{\xi}|}\,d\vec{r} = \frac{4\pi Z}{G^2}e^{i\vec{G}\cdot\vec{\xi}}.
    \end{equation}

    Совершенно аналогичные рассуждения приводят нас к тому, что
    \begin{equation}
        \tilde{v}_H(\vec{G}) = \frac{4\pi\tilde{n}(\vec{G})}{G^2},\quad
        \tilde{n}(\vec{G}) = \int_\text{cell} n(\vec{r}) e^{-i\vec{G}\cdot\vec{r}}\,d\vec{r}
    \end{equation}

    В силу сложности вида обменно-корреляционного потенциала, Фурье-коэффициенты придётся считать численно:
    \begin{equation}
        \tilde{\eps}_{xc}(\vec{G}) = \int_\text{cell} \eps_{xc}(n;\vec{r})e^{-i\vec{G}\cdot\vec{r}}\,d\vec{r}
    \end{equation}

    \begin{thebibliography}{99}
        \bibitem{Hohenberg-Kohn} \textbf{P. Hohenberg and W. Kohn} Inhomogeneous Electron Gas // Phys. Rev.~--- V. 136.~--- P. B864.
    \end{thebibliography}

    \appendix
    \section{Преобразование Фурье и уравнение Лапласа}
    Рассмотрим интеграл
    \begin{equation}
        f(\vec{k}) = \int_{\mathbb{R}^3} \frac{e^{-i\vec{k}\cdot\vec{r}}}{r}\,d\vec{r}.
    \end{equation}
    Проще всего его не брать вовсе, а вспомнить, что функция \( u = 1/r \) является решением уравнения Лапласа
    \begin{equation}
        \Delta u = -4\pi\delta(\vec{r}).
    \end{equation}
    Выполним преобразование Фурье и получим
    \begin{equation}
        -k^2 f(\vec{k}) = -4\pi,\quad f(\vec{k}) = \frac{4\pi}{k^2}.
    \end{equation}
\end{document}
