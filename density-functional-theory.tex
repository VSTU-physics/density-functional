\documentclass[article]{ncc}
\usepackage[T2A]{fontenc}
\usepackage[utf8]{inputenc}
\usepackage[russian]{babel}
\usepackage{amsmath}
\usepackage{physics}
\usepackage[square, numbers, sort&compress]{natbib}
\usepackage[colorlinks]{hyperref}
\newcommand{\eps}{\varepsilon}
\renewcommand{\phi}{\varphi}
\renewcommand{\vec}{\boldsymbol}
\title{Теория функционала плотности}
\author{Абдрахманов В.}
\begin{document}
    \maketitle
    \tableofcontents
    \section*{Обозначения}
    В этом тексте используется атомная система единиц: $ \hbar = e = m_e = 1$
    \begin{description}
        \item[$n(\vec{r})$] электронная плотность
    \end{description}
    \section{Основные положения}
    \subsection{Теория Хоэнберга-Кона}
    В своей работе \cite{Hohenberg-Kohn} сформулировали 2 теоремы, на которых построена вся теория функционала плотности.

    \begin{theorem}
        Для любой системы, находящейся во внешнем потенциале \( V(\vec{r}) \), этот внешний потенциал является (однозначным) функционалом электронной плотности основного состояния.
    \end{theorem}

    \begin{theorem}
        Энергия системы является функционалом электронной плотности. Энергия основного состояния является наименьшим значением этого функционала.
    \end{theorem}

    См. также:
    \begin{enumerate}
        \item \url{http://www.physics.metu.edu.tr/~hande/teaching/741-lectures/lecture-06.pdf}
    \end{enumerate}
    \subsection{Уравнения Кона-Шема}
    Согласно теории Хоэнберга-Кона, полная энергия системы является функционалом электронной плотности:
    \begin{equation}
        E[n] = T[n] + U_{ee}[n] + V[n].
    \end{equation}
    Основная идея состоит в представлении кинетической энергии в виде суммы:
    \begin{equation}
        T[n] = T_s[n] + T_c[n],
    \end{equation}
    где \( T_s \) -- это кинетическая энергия системы невзаимодействующих электронов, а \( T_c \) содержит корреляционную часть кинетической энергии. Для \( T_s \) мы можем в явном виде выписать выражение
    \begin{equation}
        T_s = \int \Psi^*(\vec{r}_1, \ldots, \vec{r}_N) \left(-\frac{\nabla^2}{2}\right) \Psi(\vec{r}_1, \ldots, \vec{r}_N)\,d^3\vec{r}.
    \end{equation}
    Многоэлектронная волновая функция имеет вид
    \begin{equation}
        \Psi(\vec{r}_1, \ldots, \vec{r}_N) = \frac{1}{\sqrt{N!}}\det \phi_i(\vec{r}_j),
    \end{equation}
    откуда
    \begin{equation}
        T_s = \sum_{i}^N\expval{-\frac{\nabla^2}{2}}{\phi_i},
    \end{equation}
    \begin{equation}
        n(\vec{r}) = \sum_{i}^N|\phi_i(\vec{r})|^2.
    \end{equation}
    Функционал взаимодействия электронов можно представить в виде
    \begin{equation}
        U_{ee} = V_H + V_x,
    \end{equation}
    \begin{equation}
        V_H = \frac{1}{2}\iint \frac{n(\vec{r})n(\vec{r}')}{|\vec{r}' - \vec{r}|}\,d^3\vec{r}\,d^3\vec{r}',
    \end{equation}
    Для учёта корреляции и обмена не существует нормального выражения, поэтому функционал энергии представляют в виде
    \begin{equation}
        E[n] = T_s[n] + V[n] + V_H[n] + E_{xc}[n],
    \end{equation}
    где последнее слагаемое учитывает обмено-корреляционные эффекты и выбирается от балды.
    
    Минимизировать непосредственно функционал можно, но не очень удобно. Поэтому проварьируем его и получим уравнения Кона-Шема\footnote{Как математически корректно сделать переход? Я не очень хорошо понимаю метод множителей Лагранжа}
    \begin{equation}
        \left(-\frac{\nabla^2}{2} + v(\vec{r}) + v_H(n;\vec{r}) + \eps_{xc}(n;\vec{r})\right)\phi = \eps\phi
    \end{equation}
    \begin{thebibliography}{99}
        \bibitem{Hohenberg-Kohn} \textbf{P. Hohenberg and W. Kohn} Inhomogeneous Electron Gas // Phys. Rev. ~--- V. 136. ~--- P. B864.
    \end{thebibliography}
\end{document}